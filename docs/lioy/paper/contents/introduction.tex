%!TEX root = <../main.tex>

\section{Introduction}
\label{sec:introduction}
\PARstart{M}{odern} botnets rely on Automatically Generated Domains (AGDs) to build a resilient
command and control communication infrastructure between the infected machines and the controller.
Thousands of domain names are randomly generated by the machines, using an unpredictable seed (e.g.
Twitter top tweets), which makes unfeasible any possible forecast, even if the
random generator methodology is known.
The botmaster produces the domain names using the same algorithm and seed,
and temporarily registers just one or a few of them. Therefore there will be just a few if not
only one non NXDOMAIN DNS answer that will resolve the domain name to its IP address, the
botmaster IP address. Once the command sequence has been sent over the network, nothing is left in
the hands of the defenders.

As this technique produces a high volume of NXDOMAIN DSN traffic, previous works
focused their effort in this direction. However the deployment and testing in such
cases impose quite constraining requirements, mainly related to privacy issues.

Bilge \emph{et al.} \cite{Exposure} proposed \textsc{Exposure}, which is an analysis
technique that exploits malicious DNS traffic peculiarities via passive monitoring. The main
issue arises from the use of local DNS traffic, with all its privacy-related restrictions.

Sandeep \emph{et al.}\cite{Sandeep2010} were the first ones to explore the issue of AGDs,
producing a \emph{supervised} learning based tool that leverages the randomization of AGD names as opposed
to the lower entropy featured by Human Generated Domains (HGDs).

Schiavoni \emph{et al.}\cite{Lorenzo2013} introduced \textsc{Phoenix}, an \emph{unsupervised}
system to detect, fingerprint and label malicious AGDs, based on linguistic features,
analyzing the top-hierarchy DNS traffic via passive monitoring.

We want to start from \textsc{Phoenix}, exploring a few possible further developments.
The remainder of this document is structured as follows: Section~\ref{sec:previous_work}
explains how \textsc{Phoenix} works whilst in Section~\ref{sec:proposed_approach} we
propose our future contributions and how they could improve the previous obtained results.


% section introduction (end)
