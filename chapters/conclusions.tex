%!TEX root = ../thesis_polimi.tex


\chapter{Conclusions} % (fold)
\label{chap:conclusions}
\start{C}{erberus} results seem promising and it is therefore worth continuing to
follow this approach. In this chapter we first address the \emph{limitations}
\thesystem suffers from, mainly related to the \important{Detection Phase}. Then
we will discuss the possible further developments that could fix the aforementioned
issues and enhance the system. Finally we will try to briefly summarize what we
achieved with \thesystem.

\paragraph{Chapter Organization} The remainder of the chapter is organized in
the following fashion:
\begin{itemize}
    \item in Section~\ref{sec:limitations} we discuss the current limitations of
        \thesystem;
    \item in Section~\ref{sec:future_works} we discuss how to possibly address these
        limitations and how to enhance \thesystem presenting a few further developments.
\end{itemize}

\newpage

\section{Limitations} % (fold)
\label{sec:limitations}
\sectionstart{C}{erberus} suffers from two main limitations, both related to the
\important{Detection Phase}. The first one is encountered when an unseen domain $d$
does not belong to any of the clusters the SVM was trained on (i.e. the clusters sharing their IP address with $d$): \thesystem will anyway assign it to the most
``similar'' cluster. This is
an issue that needs to be solved. We need to find a test that allows us to tell
when a single domain does not belong to any of the clusters selected to train
the SVM. If this is the case we could design a temporary buffer where these
domains are stored, and perform the clustering routine on this buffer every
$\gamma$ time to isolate new clusters.
\begin{figure}[!htp]
    \begin{minipage}{.5\textwidth}
    \centering
    \begin{tabular}{ll}
        \multicolumn{2}{l}{\textsc{Cluster 3e774}} \\
        \midrule
        \texttt{16542.com} & \texttt{79581.com} \\
        \texttt{44962.com} & \texttt{46096.com} \\
        \texttt{72879.com} & \texttt{09941.com} \\
        \texttt{42661.com} & \texttt{90338.com} \\
        \texttt{47176.com} & \texttt{69313.com} \\
        \texttt{75827.com} & \texttt{34171.com} \\
    \end{tabular}\\
    \vspace{.3cm}
    {\sffamily (a)}
    \end{minipage}%
    \begin{minipage}{.5\textwidth}
        \centering
        \begin{tabular}{l}
            \textsc{Labeled 3e774} \\
            \midrule
            \texttt{asdfuh982hdodjc.com}  \\
            \texttt{open-932978.com} \\
            \texttt{hp21821867626.mygateway.net}  \\
            \texttt{www2.h3xa.com} \\
            \texttt{061107dd0208.agulhal.com} \\
             \texttt{clkh71yhks66.com} \\
        \end{tabular}\\
        \vspace{.3cm}
        {\sffamily (b)}
    \end{minipage}
    \caption{Samples from cluster 3e774 and misclassified unseen domains.}
    \label{fig:misclassified}
\end{figure}
This situation is depicted Figure~\ref{fig:misclassified}: Figure~\ref{fig:misclassified}~(a) reports sample domains from the
cluster \textsc{3e774}, generated during the \important{Bootstrap Phase}. These
domains share an easily identifiable pattern: They all count five digits and
exhibit the \texttt{.com} TLD. In Figure~\ref{fig:misclassified}~(b) there are
a few domains labeled as \textsc{3e774}, while it is clear that they were
not produced using the same DGA.

The second major limitation regards the clustering routine. It could be the case
that the attacker's servers span across more than one Autonomous System: in our test in the wild there was just one recorded case, still it is not a phenomenon to be ignored. E.g.,
in Table~\ref{tab:flat_domains} is reported a cluster of domains found by clustering
on the flattened list of domains related to the suspicious IPs instead of grouping
them by Autonomous System. We did not follow this approach in \thesystem because
even though it is possible to find a minor number of clusters otherwise untraceable,
the overall clustering quality is seriously worsened.

\begin{table}[!htp]
    \centering
    \begin{tabular}{ll}
        \toprule
        \textsc{Domains} & \textsc{IP} \\
        \midrule
        \texttt{j5.kwu.rcvn.biz} & \texttt{64.191.8.11} \\
        \texttt{1a.flw.rcvn.biz} & \texttt{184.22.158.49} \\
        \texttt{vn.oxw.rcvn.biz} & \texttt{173.212.224.158} \\
        \texttt{sp.ebx.rcvn.biz} & \texttt{37.1.217.136} \\
        \texttt{nm.iwk.rcvn.biz} & \texttt{64.191.8.32} \\
        \texttt{41.ajs.rcvn.biz} & \texttt{184.82.98.22} \\
        \texttt{a3.fxw.rcvn.biz} & \texttt{64.191.58.73} \\
        \texttt{70.axw.rcvn.biz} & \texttt{96.9.139.111} \\
        \texttt{df.wek.rcvn.biz} & \texttt{66.96.248.248} \\
        \bottomrule
    \end{tabular}
    \caption{Cluster found without grouping by AS.}
    \label{tab:flat_domains}
\end{table}


% section limitations (end)

\section{Future Works} % (fold)
\label{sec:future_works}
\sectionstart{W}{e} start from the limitations listed in the previous section to
elicit the future developments of \thesystem. The first and crucial improvement
to be done is to find a way to determine whether a single domain is coherent with
the cluster it is about to be put into. This is crucial because, as stated in the
previous section, an IP address could serve more than one malicious activity, each
one possibly employing a different DGA. The second crucial development concerns the
clustering routine, as \thesystem is not able of detecting malicious activities that
span different Autonomous Systems. One possible way to address this issue is to
combine the clustering results obtained when grouping and when not grouping by AS.

One improvement regards the Classifier. \citet{shafer2008} proposed the
\emph{Conformal Predictor}, a tool that tells you how much confident you can be
about a prediction, based on the history of your past predictions. We would like this
component to be added to \thesystem, as we would have a way to determine how much
confident we can be when labeling an unseen domain to \thesystem's ground truth.

One important future development is to save the SVMs as they are trained, storing
a \emph{dictionary} where the keys are the clusters the SVM was trained on and the
values are the SVMs objects themselves.

Finally we would like to deploy \thesystem and analyze real time data. Then
we could develop a web application where the user can \emph{i)} ask the system
whether a domain or an IP is malicious (i.e., that domain or IP is in one of the
clusters of the ground truth), \emph{ii)} download the ground truth
and \emph{iii)} upload a dump of traffic (think of a corporate network) and have
\thesystem analyzing it to \emph{detect} and \emph{discover} malicious activities.
% section future_works (end)


% section limitations (end)

\section{Concluding Remarks} % (fold)
\label{sec:conclusions}
We presented \thesystem, a system able to detect and characterize unseen DGA-based
threats, botnets above all, easily deployable and free of privacy related issues
as it analyzes passive DNS data, and able to operate without any \emph{a priori}
knowledge. We started from and improved the results obtained by
\phoenix~\cite{schiavoni2013}, being able to discover new threats that rely on C\&C
servers that exhibit unseen IP addresses, and building a better classifier
which is now trained not on \emph{ad hoc} parameters, but leverages the domain
name itself as classification feature.

We tested \thesystem in the wild, analyzing one week of passive DNS data collected
from the 7th to the 14th of February, 2013. The system, without any \emph{a priori} knowledge, was able to isolate
47 clusters of malicious domains belonging to various threats among which we find the well known \texttt{Palevo}, \texttt{Sality} and \texttt{Jadtre}.
Moreover we tested the classifier in a four classes classification test, featuring an overall
precision of 93\% with 800 points in the training set.

Therefore, given the results obtained, we argue that \thesystem improves the systems proposed so far by adopting an \emph{unsupervised} approach and analyzing
passive DNS data, two features that allow \thesystem to automatically detect
previously unseen DGA-based threats in the Internet.
% section conclusions (end)

% chapter conclusions (end)