%!TEX root = ../thesis_polimi.tex

\chapter{Phoenix} % (fold)
\label{chap:phoenix}

\start{I}{n this chapter} we shall present \phoenix, a system aiming at finding,
characterizing and tracking domain generation algorithms from passive DNS
monitoring. \phoenix is the result of Stefano Schiavoni's Master's
Thesis~\cite{schiavoni2013}. \\
As \phoenix is the starting point and the core of \thesystem, it is mandatory
to have a brief, yet thorough summary of the system.

We will explain how \phoenix separates domains that are likely to be
automatically generated (AGDs) from those that are likely to be created by
humans (HGDs). Then we will explain how it successfully groups these domains,
finding families of Domain Generation Algorithms and their relative Command
\& Control servers.

\paragraph{Chapter Organization} The remainder of the chapter is organized in
the following fashion:
\begin{itemize}
    \item in Section~\ref{sec:goals_and_challenges} we will explain what \phoenix
        aims to achieve and the difficulties Schiavoni had to cope with;
    \item in Section~\ref{sec:system_description} we describe how \phoenix was
        designed and implemented;
    \item in Section~\ref{sec:system_evaluation} we will present \phoenix results,
        analyzing its shortcomints, starting point for \thesystem.
\end{itemize}

\section{Goals and Challenges} % (fold)
\label{sec:goals_and_challenges}

% section goals_and_challenges (end)

\section{System Description} % (fold)
\label{sec:system_description}
\sectionstart{P}{hoenix} is organized into two main dependent modules: the
\component{DGA Discovery} module and the \component{DGA Detection} module. The first
one is in charge of \emph{discovering} domains that relate to the botnet threat
and cluster together those domains that refer to the same family of DGAs. The latter
uses the knowledge produced by the first one to classify unseen domains as malicious or benign, and, in the first case, to assign them to one of the clusters created
during the first phase. In the following of this section we shall further analyze how \phoenix
achieves its goals.

\begin{figure}[!htp]
    \centering
    \begin{tikzpicture}[node distance=2cm]
        \node (FEED) {Malicious Domains};
        \node[component] [below=of FEED.south, anchor=north] (DISCOVERY) {DGA Discovery};
        \node[component] [right=of DISCOVERY.east] (DETECTION) {DGA Detection};

        \path[->, draw]
            (FEED) -- (DISCOVERY)
            (DISCOVERY) -- (DETECTION);
    \end{tikzpicture}
    \caption{\phoenix high level system architecture.}
    \label{fig:phoenix_architecture}
\end{figure}

\subsection{DGA Discovery} % (fold)
\label{sub:dga_discovery}
The goal of the DGA Discovery module is to find clusters of malicious domains
likely to be automatically generated that resolve to the same (set of) IP. Therefore
we have to go through two subsequent stages, filtering and clustering. The first phase
is achieved leveraging four linguistic features that should distinguish \emph{words}
from \emph{random strings}. The second phase employs a Hierarchical Spectral
clustering algorithm to extract clusters of malicious AGDs from the list produced by
the previous stage. These clusters are likely to correspond to families of DGAs, and
evidence shall prove our hypothesis.

\begin{figure}[!htp]
    \centering
    \begin{tikzpicture}[node distance=2cm]
        \node (FEED) {Malicious Domains};
        \node[component] [below=of FEED.south, anchor=north] (FILTERING) {DGA Filtering};
        \node[component] [right=of FILTERING.east] (CLUSTERING) {Clustering};

        \path[->, draw]
            (FEED) -- (FILTERING)
            (FILTERING) -- (CLUSTERING);
    \end{tikzpicture}
    \caption{\phoenix DGA Discovery architecture}
    \label{fig:discovery_architecture}
\end{figure}

\subsubsection{DGA Filtering} % (fold)
\label{ssub:dga_filtering}
The goal of this stage is to separate domains that are Automatically Generated from
those that are created by humans.
It is quite easy for a human being to tell whether a domain falls into the first or
the second category: Whilst HGDs are meant to be easy to remember and readable, AGDs
look like \emph{high-entropy} strings, as in Table~\ref{tab:agd_sample}.
\citet{schiavoni2013} developed an algorithm to capture these features and separate
the domains in an automatic fashions.

\begin{table}[!htp]
    \centering
    \begin{tabular}{lll}
        \verb+vljiic.org+           & \verb+vitgyyizzz.biz+ & \verb+79ec8...f57ef.co.cc+ \\
        \verb+f0938...772fb.co.cc+  & \verb+nlgie.org+      & \verb+gkeqr.org+ \\
        \verb+jyzirvf.info+         & \verb+aawrqv.biz+     & \verb+xtknjczaafo.biz+ \\
        \verb+hughfgh142.tk+        & \verb+yxipat.cn+      & \verb+yxzje.info+ \\
        \verb+fyivbrl3b0dyf.cn+     & \verb+rboed.info+     & \verb+ukujhjg11.tk+ \\
    \end{tabular}
    \caption{Samples of AGDs.}
    \label{tab:agd_sample}
\end{table}

\begin{figure}[!htp]
    \begin{minipage}{0.5\linewidth}
        \centering
        $$ d = \mathtt{facebook.com}$$\vspace{-0.2cm}
        $$R(d) = \frac{|\mathtt{face}| +|\mathtt{book}|}{|\mathtt{facebook}|} = 1$$
        \vspace{0.3cm} \\ likely \bfseries{HGD}
        \end{minipage}\begin{minipage}{0.5\linewidth}
        \centering
        $$ d = \mathtt{pub03str.info}$$\vspace{-0.2cm}
        $$R(d) = \frac{|\mathtt{pub}|}{|\mathtt{pub03str}|} = 0.375.$$
        \vspace{0.3cm} \\ likely \bfseries{AGD}
    \end{minipage}
    \caption{Meaningful Word Ratio example, \citet{schiavoni2013}}
    \label{fig:words_ratio}
\end{figure}

\begin{figure}[!htp]
\begin{minipage}{0.5\textwidth}
    \centering
    $$ d = \mathtt{facebook.com}$$\vspace{-0.2cm}
    \begin{scriptsize}
        \begin{tabular}{c c c c c c c}
          \texttt{fa} & \texttt{ac} & \texttt{ce} & \texttt{eb} & \texttt{bo}  & \texttt{oo}  & \texttt{ok}\\
          109 & 343 & 438 & 29 & 118 & 114 & 45
        \end{tabular}\end{scriptsize}\\
        \vspace{0.5cm}
        mean: $S_2 = 170.8$
        \vspace{0.3cm} \\ likely \textbf{HGD}
\end{minipage}%
\begin{minipage}{0.5\textwidth}
        \centering
        $$ d = \mathtt{aawrqv.com}$$\vspace{-0.2cm}\\
        \begin{scriptsize}
        \begin{tabular}{ c c c c c }
          \texttt{aa} & \texttt{aw} & \texttt{wr} & \texttt{rq} & \texttt{qv}\\
          4 & 45 & 17 & 0 & 0
        \end{tabular}\end{scriptsize}\\
        \vspace{0.5cm}
        mean: $S_2 = 13.2$
        \vspace{0.3cm} \\ likely \textbf{AGD}
\end{minipage}
\caption{2-gram score example, \citet{schiavoni2013}}.
\end{figure}

\begin{equation} \label{eq:feature_vector}
    f\left(d\right) = \left[R\left(d\right), S_1\left(d\right),
    S_2\left(d\right), S_3\left(d\right)\right]^T
\end{equation}

% subsubsection dga_filtering (end)

\subsubsection{DGA Clustering} % (fold)
\label{ssub:dga_clustering}

% subsubsection dga_clustering (end)

% subsection dga_discovery (end)

\subsection{DGA Detection} % (fold)
\label{sub:dga_detection}

% subsection dga_detection (end)

% section system_description (end)

\section{System Evaluation} % (fold)
\label{sec:system_evaluation}

% section system_evaluation (end)

\section{Shortcomings} % (fold)
\label{sec:shortcomings}

% section shortcomings (end)


\section{Summary} % (fold)
\label{sec:summary}

% section summary (end)

% chapter phoenix (end)