%!TEX root = ../thesis_polimi.tex


\chapter*{Abstract}
\start{B}{otnets}  are networks of infected machines (the \emph{bots}) controlled
by an external entity, the \emph{botmaster}, who uses this infrastructure to
carry out malicious activities, e.g., spamming and Distributed Denial of Service. The Command and Control Server (C\&C) is the machine
employed by the botmaster to dispatch orders to and gather data from the bots,
and the communication is established through a variety of distributed or
centralized protocols, which can vary from botnet to botnet. In the case of
DGA-based botnets, a Domain Generation Algorithm (DGA) is used to find the
\emph{rendezvous} point between the \emph{bots} and the \emph{botmaster}.
Botnets represent one of the most widespread and dangerous threats on the Internet and
therefore it is natural that researchers from both the industry and the academia
are striving to mitigate this phenomenon. The mitigation of a botnet
is a topic widely covered in literature, where we find many works that propose approaches
for its detection. Still, all of these systems suffer from the major
shortcomings of either using a supervised approach, which means that the system
needs some \emph{a priori} knowledge, or leveraging DNS data containing
information on the infected machines, which leads to issues related to the
users' privacy and the deployment of such systems.

We propose \thesystem, an automated system based on machine learning, capable to automatically discover new botnets and use this
knowledge to detect and characterize malicious activities. \thesystem analyzes passive
DNS data, free of any privacy issues, which allows the system to be easily
deployable, and uses an unsupervised approach, i.e., \thesystem needs no
\emph{a priori} knowledge. In fact the system applies a series of filters to
discard legitimate domains while keeping domains generated by AGDs and likely to be malicious. Then, \thesystem
keeps record of the activity related to the IP addresses of those domains, and,
after $\Delta$ time, it is able to isolate clusters of domains belonging to the same
malicious activity. This knowledge is later used to train a classifier that will analyze
new DNS data for detection.

We tested our system in the wild by analyzing one week of real passive DNS data.
\thesystem was able to detect 47 new clusters of malicious activities: Well
known botnets as \texttt{Jadtre}, \texttt{Sality} and \texttt{Palevo} were found among the others.
Moreover the tests we ran on the classifier showed an overall accuracy of 93\%, proving
the effectiveness of the system.


\selectlanguage{italian}
\chapter*{Ampio Estratto}
\start{L}{e botnet} sono considerate una tra le più diffuse minacce nell'Internet.
Analisti dal mondo dell'industria e ricercatori di sicurezza informatica sono
continuamente all'opera, al fine di mitigare questo fenomeno in perenne crescita.
Recenti rapporti redatti da McAfee~\cite{mcafee2013} ed Enisa~\cite{enisa2013}
confermano come tutte le attività malevoli presenti nell'Internet non mostrino il
minimo segno di scostamento dal loro continuativo tasso di crescita, botnet comprese.
Gli \emph{attacker} inoltre, colpiscono sempre più tipologie di piattaforme:
nell'aprile 2012 più di 600,000 computer Apple~\cite{enisa2012} furono infettati
dal \emph{malware} conosciuto come \texttt{Flashback}.
Inoltre, le minacce riguardanti piattaforme mobili, hanno visto un'impressionante
crescita, diventando un serio problema per gli utilizzatori di \emph{smartphone}
Android, il sistema operativo più colpito.

Una ricerca condotta da \citet{grier2012manufacturing} ha mostrato come
le attività malevole non siano più perpetrate dagli attaccanti a scopo ludico, ma
per fini lucrativi. Il sorgere del modello \emph{exploit-as-a-service} per
portare a compimento la compromissione del \emph{browser} tramite attacco
\emph{drive-by-download} è un fenomeno preoccupante, in cui i kit di exploit sono
venduti al pari di un qualsiasi altro prodotto per le necessità degli attaccanti.

Le stesse botnet sono ormai infrastrutture concepite esplicitamente per fini di lucro.
\texttt{Torpig} e \texttt{Zeus} ad esempio, mirano a rubare credenziali di accesso
a conti corrente dalle macchine infette e inviarli agli attaccanti. Un altro
tipo di malware lucrativo prende il nome di \emph{ransomware}.
Il \emph{ransomware} non è una nuova minaccia, tuttavia è tornata prepotentemente
in scena con l'avvento di \texttt{Cryptolocker}, un \emph{malware} capace di
collezionare guadagni che sono stati conservativamente stimati pari a
1,100,000 USD~\cite{spagnuolo2013}.

La mitigazione delle \emph{botnet} è pertanto di interesse primario per i \emph{difensori}, i quali a tale scopo concentrano i loro sforzi nell'individuare gli indirizzi
IP dei cosiddetti Server di Comando e Controllo (C\&C). Un Server C\&C è la
macchina dalla quale gli attaccanti inviano gli ordini e raccolgono i dati
provenienti dalle macchine infette. In un'architettura di botnet di tipo
centralizzato (la più diffusa), nel momento in cui la comunicazione tra il Server
C\&C e le macchine infette, i \emph{bots}, è terminata, questi ultimi diventano
\emph{dormienti} e solitamente innocui.

Gli attaccanti sono consci di questa debolezza nella loro infrastruttura, e hanno
sviluppato tecniche sempre più sofisticate al fine di rendere l'interruzione del
canale di comunicazione C\&C uno sforzo considerevole per i difensori.
A tale scopo, la maggior parte della comunicazione delle moderne botnet,
si appoggia sull'utilizzo dei Domain Generation Algorithm (DGA), che rendono
il punto di \emph{rendezvous} non prevedibile e il canale di comunicazione resistente
alle interruzioni. Questi algoritmi generano liste di domini casuali ogni $\Delta$
tempo, ad esempio un giorno, a volte utilizzando semi imprevedibili, domini che i \emph{bot}
tentano di contattare tramite richieste di tipo HTTP. L'attaccante registra solo un dominio e aspetta che uno dei \emph{bot} contatti l'URL corretto. Quando ciò accade,
la comunicazione ha inizio e i dati possono essere trasmessi in entrambe le direzioni.

La mitigazione delle botnet è un argomento ampiamente discusso in letteratura,
in cui vari autori hanno proposto strumenti di individuazione i quali, seppur
efficaci, si appoggiano principalmente su un approccio di tipo \emph{supervisionato}
o sugli indirizzi IP degli utenti o su entrambi: due aspetti che consideriamo come
difetti. Infatti, l'approccio supervisionato richiede che vengano forniti
dati etichettati in ingresso al sistema. Ciò comporta la necessità di avere un altro
modo per individuare la minaccia ed etichettarla, prima che il sistema possa essere
operativo. Inoltre questo approccio risulta minacciato da attività malevole che
mostrano caratteristiche prima non considerate. Utilizzare l'indirizzo IP degli
utenti è considerato un difetto poiché comporta problemi relativi alla privacy
degli utenti stessi e al dislocamento di un sistema di monitoraggio a livelli bassi
della gerarchia DNS.

Nel 2013, \citet{schiavoni2013} proposero \phoenix, un sistema di rivelazione che
soddisfa la maggior parte dei nostri criteri, poiché non richiede conoscenza
\emph{a priori} e analizza dati DNS passivi, franchi di qualsiasi problema legato
alla privacy. \phoenix è in grado di \emph{scoprire} cluster di domini utilizzati
per la comunicazione dei Server C\&C, svelando in tal modo i loro indirizzi IP,
e di \emph{individuare} domini malevoli, utilizzando la conoscenza prodotta
precedentemente. Nonostante i notevoli risultati, \phoenix non è in grado di
individuare \emph{i)} minacce che utilizzano Server C\&C con IP prima non visti,
\emph{ii)} minacce che modificano leggermente il DGA, e.g., cambiando l'insieme di
TLD utilizzati e \emph{iii)} manca di validazione con dati reali.

Noi proponiamo \thesystem, un sistema di individuazione che migliora i risultati
raggiunti da \phoenix, dimostrandosi capace di individuare minacce non note
utilizzando un approccio di tipo non supervisionato.

\thesystem affronta tre fasi per raggiungere questo risultato. La prima fase
prende il nome di \important{Fase di Bootstrap}, in cui utilizza \phoenix
per generare \emph{ground truth} da utilizzarsi in seguito durante la
\important{Fase di Detection}. La \emph{ground truth} è composta da una lista di
\emph{cluster} di domini che fanno riferimento ad attività malevole basate su DGA,
e.g., \emph{botnet} e \emph{trojan}.

Dopo che il sistema è stato inizializzato, comincia ad analizzare
un flusso di dati DNS passivi, il quale consiste di risposte DNS: il nome di dominio,
l'indirizzo IP a cui risolve e il TTL. Questi dati devono essere filtrati dai domini
legittimi. A tale scopo, durante la \important{Fase di Filtering}, una serie di
euristiche viene applicata ai dati, euristiche che tengono conto di parametri come
la data di registrazione il TLD utilizzato e il TTL. I domini che rimangono alla
fine del processo di filtraggio sono da considerarsi ``probabilmente malevoli''.

Questi domini sono i dati di ingresso per la sovracitata \important{Fase di Detection}, in cui proviamo a classificare tali dati utilizzando la conoscenza prodotta in
\important{Fase di Bootstrap}. A tale scopo, \thesystem guarda se il dominio
sconosciuto $d$ condivide il proprio indirizzo IP con uno dei cluster: in tal caso
alleniamo una Support Vector Machine equipaggiata con la funzione
Subsequence String Kernel~\cite{lodhi2002}, e la utilizziamo per classificare $d$.
Altrimenti iniziamo a registrare le attività legate all'indirizzo IP di $d$, $l$.
Ciò significa che teniamo traccia dei domini probabilmente malevoli che hanno
risolto su $l$ nel corso del tempo. Dopo una settimana di registrazione, \thesystem
raggruppa questi indirizzi IP sospetti a seconda dell'Autonomous System a cui
appartengono ed esegue una routine di clustering basata sull'algoritmo DBSCAN e
il Subsequence String Kernel come misura di distanza. Questi cluster sono poi
aggiunti alla \emph{ground truth} e la conoscenza ora aumentata viene utilizzata
per analizzare nuovi dati DNS.

Abbiamo eseguito un esperimento utilizzando dati passivi DNS reali,
raccolti da una macchina ISC/SIE nell'arco di una settimana. In questo periodo, \thesystem ha
classificato 187 domini malevoli, di cui 167 appartenevano alla minaccia chiamata
\texttt{Conficker}, precedentemente scoperta da \phoenix. Dopo sette giorni
avevamo raccolto 1,300 indirizzi IP sospetti, per un totale di 3,576 domini.
Quindi la routine di clustering è stata in grado di produrre 47 nuovi cluster,
i quali esibivano indirizzi IP noti, appartenenti a minacce come \texttt{Palevo}
e \texttt{Sality}. I cluster sono stati aggiunti alla precedente conoscenza e il
giorno seguente \thesystem è stato in grado di individuare 319 domini
malevoli.

\paragraph{Organizzazione del Documento}
Il resto del documento è organizzato nel modo seguente:
\begin{itemize}
    \item nel Capitolo~\ref{chap:botnets} forniremo le informazioni necessarie a comprendere il problema in analisi;
    \item nel Capitolo~\ref{chap:motivation} approfondiremo le motivazioni per cui
        abbiamo interesse a studiare questo fenomeno;
    \item nel Capitolo~\ref{chap:approach} presenteremo \thesystem, un
        sistema di rivelazione di attività malevole basate su DGA;
    \item nel Capitolo~\ref{chap:implementation} spiegheremo come \thesystem
        è stato implementato e le tecnologie impiegate;
    \item nel Capitolo~\ref{chap:validation} verificheremo l'efficacia di
    \thesystem;
    \item nel Capitolo~\ref{chap:conclusions} analizzeremo i difetti di
    \thesystem e proporremo alcuni possibili rimedi, assieme a sviluppi futuri.
\end{itemize}

\selectlanguage{english}
